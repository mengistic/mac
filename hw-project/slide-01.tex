\documentclass{beamer}
% ==> preamble

% ==> mac template
%https://www.r-bloggers.com/2011/11/create-your-own-beamer-template/
%\usepackage{pgfcomp-version-0-65}
\usepackage{pgf}
\usepackage{tikz}
\usepackage{amsmath}
\linespread{1.25}
\usepackage{multicol}


% ==> math

\usepackage{mathtools}
\mathtoolsset{centercolon} % not work when using |mathpazo|
\DeclarePairedDelimiter\abs{\lvert}{\rvert}
\DeclarePairedDelimiterX\norm[1]\lVert\rVert{
	\ifblank{#1}{\:\cdot\:}{#1}
}
\def\set#1#2{\left\{#1 ~:~ #2\right\}}
\def\permil{\text{\hskip 0.3pt\englishfont\textperthousand}}


\DeclareMathOperator{\arccot}{cot}
\DeclareMathOperator{\arcsec}{arcsec}
\DeclareMathOperator{\arccsc}{arccsc}
\DeclareMathOperator{\lcm}{lcm}
\DeclareMathOperator{\ord}{ord}
\DeclareMathOperator{\sym}{sym}
\DeclareMathOperator{\tr}{trace}
\DeclareMathOperator{\spans}{span}
\DeclareMathOperator{\rank}{rank}
\DeclareMathOperator{\col}{col}
\DeclareMathOperator{\row}{row}
\DeclareMathOperator{\sign}{sign}
\DeclareMathOperator{\perm}{perm}
\DeclareMathOperator{\coef}{coef}
%\DeclareMathOperator{\proj}{proj}

\def\N{\mathbb N}
\def\Z{\mathbb Z}
\def\Q{\mathbb Q}
\def\R{\mathbb R}
\def\C{\mathbb C}
\def\F{\mathbb F}
\def\P{\mathbb P}

\def\L{\mathcal L}
\def\U{\mathcal U}

\def\labelitemi{$\circ$}
\def\inv{^{-1}}
\def\ang#1{\left\langle#1\right\rangle}
\def\tran{^\mathrm{T}}
\renewcommand{\vec}[1]{\mathbf{#1}}
\def\perc{^\perp}
\def\proj#1#2{\mathrm{proj}_{#2}{#1}}



\usepackage{bbold}
\let\altmathbb\mathbb
\AtBeginDocument{\let\mathbb\altmathbb}
% <==
% ==> color
\usepackage{color}

\definecolor{Ftitle}{RGB}{20, 0, 200}
\definecolor{Ftext}{RGB}{0,0,0}
\definecolor{Fenum}{RGB}{20, 70, 100}

\definecolor{StdTitle}{RGB}{26, 33, 141}
\definecolor{Descitem}{RGB}{20, 70, 100}
%\definecolor{StdTitle}{RGB}{20, 70, 100}
\definecolor{StdBody}{RGB}{213,24,0}

\definecolor{AlTitle}{RGB}{255, 190, 190}
\definecolor{AlBody}{RGB}{213,24,0}

\definecolor{ExTitle}{RGB}{201, 217, 217}
\definecolor{ExBody}{RGB}{213,24,0}


\setbeamercolor{normal text}{fg = Ftext}
\setbeamercolor{frametitle}{fg = Ftitle}
\setbeamercolor{title}{fg = Ftitle}
\setbeamercolor{section in toc}{fg = Ftitle}
\setbeamercolor{section in toc shaded}{fg = Ftitle}
\setbeamercolor{item}{fg = Fenum}
\setbeamercolor{subitem}{fg = Fenum}
\setbeamercolor{subsubitem}{fg = Ftitle}
\setbeamercolor{description item}{fg = Descitem}
\setbeamercolor{caption}{fg = Ftitle}
\setbeamercolor{caption name}{fg = Ftitle}

% Standard block
\setbeamercolor{block title}{fg = Descitem, bg = StdTitle!15!white}
\setbeamercolor{block body}{bg = StdBody!12!white}

% Alert block
\setbeamercolor{block title alerted}{bg = AlTitle}
\setbeamercolor{block body alerted}{bg = AlBody!5!white}

% Example block
\setbeamercolor{block title example}{bg = ExTitle}
\setbeamercolor{block body example}{bg = ExBody!5!white}

% <==
% ==> fonts
%\usefonttheme{professionalfonts}

% Font for the presentation title
\setbeamerfont{title}{size = \huge}

% Font of the frame titles
\setbeamerfont{frametitle}{size = \Large}
% <==
% ==> inner theme
\useinnertheme{circles}
\setbeamertemplate{itemize items}[circle]
\setbeamertemplate{enumerate items}[circle]
\setbeamertemplate{navigation symbols}{}

% ==> logo line
\newcommand{\LogoLine}{%
  \raisebox{-12mm}[0pt][0pt]{%
  \begin{pgfpicture}{0mm}{0mm}{0mm}{0mm}
    \pgfsetlinewidth{0.28mm}
    \color{gray!70!blue}
    \pgfline{\pgfpoint{-3 mm}{1mm}}{\pgfpoint{10.125 cm}{1mm}}
  \end{pgfpicture}}
}
\setbeamertemplate{headline}[text line]{\LogoLine}
% <==
% ==> logo
\setbeamertemplate{sidebar canvas right}{
  \vspace*{0 pt}\hspace*{-42 pt}%
  {\includegraphics[height=40 pt]{../pic/mac.jpg}}
}
% <==
% ==> adjust title
\setbeamertemplate{frametitle}{
  \vspace*{3mm}\hspace*{-2mm}\insertframetitle
}
% <==
% ==> footline
\newcommand{\Ffootline}{%
  \insertsection % The left end of the footline
  \hfill
  \textit{MAC} % The center
  \hfill
  \insertframenumber/\inserttotalframenumber % And the right end
}

\setbeamertemplate{footline}{%
  \usebeamerfont{structure}
  \begin{beamercolorbox}[
    wd=\paperwidth,ht=2.25ex,dp=1ex
    ]{title in head/foot}%
    \tiny\hspace*{4mm} \Ffootline \hspace{4mm}
  \end{beamercolorbox}
}
% <==
% <==
% ==> title
% Title

\newcommand{\TitleLine}{%
\raisebox{-12mm}[0pt][0pt]{%
\begin{pgfpicture}{0mm}{0mm}{0mm}{0mm}
\pgfsetlinewidth{0.10mm}
\color{gray}
\pgfline{\pgfpoint{55mm}{0mm}}{\pgfpoint{55mm}{50mm}}
\end{pgfpicture}}}

\newcommand{\MyTitle}{%
\hspace*{60mm}\vspace{-25mm}
\centering \inserttitle}

% Subtitle
\newcommand{\MySubTitle}{%
\hspace*{60mm}\vspace{-25mm}
\centering \footnotesize \textit{\insertsubtitle}}

% Author
\newcommand{\MyAuthor}{
\hspace*{60mm}\vspace{-25mm}
\centering \insertauthor}

% Institute
\newcommand{\MyInstitute}{
\hspace*{60mm}\vspace{-25mm}
\centering \footnotesize \textit{\insertinstitute}}

% Date
\newcommand{\MyDate}{
\hspace*{60mm}\vspace{-25mm}
\centering \insertdate}



% We declare the image that will be used as the logo
\pgfdeclareimage[width = 0.20\paperwidth]{big}{../pic/mac.jpg}

\setbeamertemplate{title page}{\TitleLine
\hspace*{11mm}\vspace*{-60mm}
\begin{beamercolorbox}[wd=0.5\paperwidth,ht=0.13\paperwidth]{Title}
\pgfuseimage{big}
\end{beamercolorbox}
%
\begin{beamercolorbox}[wd=\paperwidth,ht=0.06\paperwidth]{Title}
\usebeamerfont{Title}%
\MyTitle
\end{beamercolorbox}
%
\begin{beamercolorbox}[wd=\paperwidth,ht=0.03\paperwidth]{Title}
\usebeamerfont{Title}%
\MySubTitle
\end{beamercolorbox}
%
\begin{beamercolorbox}[wd=\paperwidth,ht=0.06\paperwidth]{Title}
\usebeamerfont{Title}%
\MyAuthor
\end{beamercolorbox}
%
\begin{beamercolorbox}[wd=\paperwidth,ht=0.03\paperwidth]{Title}
\usebeamerfont{head/foot}%
\MyInstitute
\end{beamercolorbox}
%
\begin{beamercolorbox}[wd=\paperwidth,ht=0.07\paperwidth]{Title}
\usebeamerfont{Title}%
\MyDate
\end{beamercolorbox}}
% <==
% <==

\def\Rp{\mathbb{R}^{+}}

\usepackage{multicol}
\title{Properties of Real Numbers}
\author{Jeng Inger, Hun Sivmeng}
\date{14 March 2022}
% <==


\begin{document}

% ==> title
\begin{frame}
  \maketitle
\end{frame}
\begin{frame}[b]
  \tableofcontents
\end{frame}
% <==
% ==> intro
\section{Introduction}
\begin{frame}{Discussion}
  Suppose we want to solve $x^2-2=0$ in the field $\Q$.
  It turns out that this equation has no roots in $\Q$.
  So we need to extend this field to a new one --- 
  a \emph{better} one, called the set of real numbers and
  is denoted by $\R$.\\[0.3cm]
  %
  What are the properties of this new set $\R$, and how
  do we extend this exactly? \\[0.3cm]
  %
  But before we proceed, we ask the audience to forget
  all the properties of $\R$ we learned in highschool 
  because we're going to start from the bottom up.\\[0.3cm]
  %
  First, because $\R$ is an extension of $\Q$, thus $\R$
  must inherit all the properties of $\Q$.
  %
  %We already know the set $\Q$ of rational numbers.
  %However, it turns out that not all numbers are rational
  %(take $\sqrt 2$ for instance.)\\[0.3cm]
  %%
  %So we need to extend $\Q$ to a new and bigger set.
  %We call the new set $\R$ --- the set of real numbers. 
  %What properties should this new set have? \\[0.3cm]
  %%
  %Following slides, we present the properties of $\R$.
  %First, because $\Q$ is an ordered field, so must be $\R$.
\end{frame}
% <==
% ==> algebraic
\section{Algebraic Properties}
% ==> addition
\begin{frame}{Algebraic Properties}
  The set $\R$ has the following algebraic properties:
  \begin{block}{Algebraic Properties (Addition)}
    \begin{enumerate}
      \item \textit{(closed):}
        $a+b\in\R$
      \item \textit{(commutative):}
        $a+b=b+a$
      \item \textit{(associative):}
        $a+(b+c)=(a+b)+c$
      \item \textit{(additive identity):}
        $\exists 0\in\R,\forall a\in\R:~ 0+a=a$
      \item \textit{(additive inverse):}
        $\forall a\in\R,~\exists a'\in\R:~ a+a'=0$.
    \end{enumerate}
  \end{block}
\end{frame}
% <==
% ==> multiplication
\begin{frame}
  \begin{block}{Algebraic Properties (Multiplication)}
    \begin{enumerate}
      \item \textit{(closed):}
        $ab\in\R$
      \item \textit{(commutative):}
        $ab=ba$
      \item \textit{(associative):}
        $a(bc)=(ab)c$
      \item \textit{(multiplicative identity):}
        $\exists 1\in\R,\forall a\in\R:~ a\cdot 1=a$
      \item \textit{(multiplicative inverse):}
        $\forall a\in\R,a\neq 0,~\exists a^{*}\in\R:~ aa^{*}=1$
      \item \textit{(distributive law): }
        $a(b+c)=ab+bc$.
    \end{enumerate}
  \end{block}
  Thus, $\R$ is a field with two operations, namely 
  \textit{addition} and \textit{multiplication}.
\end{frame}
% <==
% ==> application
\begin{frame}
  From this algebraic properties, we can prove the basic arithmetic
  we learned in highschool with ease. 
  \begin{lemma}
    \begin{itemize}
      \item additive identity/inverse is unique
      \item multiplicative identity/inverse is unique
      \item $x+a=y+a \iff x=y$
      \item $a\cdot 0=0$
      \item $-a=(-1)\cdot a$
      \item $ab=0 \iff a=0\text{ or }b=0$
    \end{itemize}
  \end{lemma}
\end{frame}
% <==
% <==
% ==> ordering
\section{Ordering Properties}
\begin{frame}{Ordering Properties}
  There is a non-empty subset of $\R$, called
  the set of positive real numbers which is denoted
  by $\Rp$. This set has the properties:
  %
  \begin{block}{Ordering Properties}
    \begin{itemize}
      \item if $a\in\R$, then $a\in\Rp$ or $a=0$ or $-a\in\Rp$;
      \item if $a,b\in\Rp$ then $a+b\in\Rp$;
      \item if $a,b\in\Rp$ then $ab\in\Rp$.
    \end{itemize}
  \end{block}

  We then define relations $<,~\leq,~>$ and $~\geq$  as follows: 
  \begin{itemize}
    \item $a>b$ whenever $a-b\in\Rp$;
    \item $a\geq b$ whenever $a>b$ or $a=b$;
    \item $b<a$ whenever $a>b$;
    \item $b\leq a$ whenever $a\geq b$;
  \end{itemize}
\end{frame}

\begin{frame}
  From this ordering properties, we can prove the following:
  \begin{lemma}
    Let $a,b,c\in\R$.
    \begin{itemize}
      \item then $a>b$ or $a=b$ or $a<b$.
      \item if $a>b$, then $a+c>b+c$.
      \item if $a>b$ and $b>c$, then $a>c$.
      \item if $a>b$ and $c>0$, then $ac>bc$.
      \item if $a>b$ and $c<0$, then $ac<bc$.
    \end{itemize}
  \end{lemma}
\end{frame}

% <==
% ==> completeness
% ==> bouned
\section{Completeness Property}
\begin{frame}{Completeness}
  Up until this point, the set $\R$ still behaves like $\Q$,
  namely $\R$ is an odered field just like $\Q$. How do we make
  $\R$ \emph{better} than $\Q$ ?

  \begin{definition}
    Let $A\subset\R$. We call
    \begin{itemize}
      \item $A$ is bounded above if $\exists M\in\R$ st $a\leq M$ for all $a\in A$.
      \item $A$ is bounded below if $\exists m\in\R$ st $m\leq a$ for all $a\in A$.
      \item $A$ is bounded if $A$ is both bounded above and below, i.e.
        $\exists M>0$ st $\abs{a}\leq M$ for all $a\in A$.
    \end{itemize}
  \end{definition}

  We call $M,m$ the upper bound and lower bound 
  of $A$, respectively.
  The \emph{Completeness} property of $\R$ states that
\end{frame}
% <==
% ==> AoC
\begin{frame}[b]
  \begin{block}{Axiom of Completeness -- (AoC)}
    Let $A\subset\R$. If $A$ is bounded above, 
    then $A$ has \emph{least upper bound.}
  \end{block}
  This axiom is a mathematical way to say that $\R$ has no gaps.
  It also applies to sets that are bouned below.

  \begin{corollary}
    Let $B\subset\R$. If $B$ is bouned below, 
    then $B$ has greatest lower bound.
  \end{corollary}

  We denote the 
  \begin{itemize}
    \item least upper bound of $A$ by $\sup A$ 
    \item greatest lower bound of $A$ by $\inf A$
  \end{itemize}
  
\end{frame}
% <==
% ==> application
\begin{frame}{}
  \begin{lemma}
    There is $x\in\R$ such that $x^2=2$.
  \end{lemma}
  \begin{proof}
    Consider the set $S=\{x\in\R~:~ x^2\leq 2\}$. This set is clearly
    bouned above. Hence by AoC, $S$ has supremum. Let $s=\sup S\in\R$.
    We claim that $s^2=2$. From ordering property of $\R$, one of the
    following must true
    \[s^2<2, \quad\text{or}\quad s^2>2, \quad\text{or}\quad s^2=2.\]
    %
    \textbf{Case } if $s^2<2$:
  \end{proof}
\end{frame}
% <==





% <==
% ==> why R
\section{Why do Analysis on $\R$}
\begin{frame}{Why study on $\R$}
  As we mentioned above, we can't do analysis on $\Q$
  because some numbers simply don't exist in $\Q$.\\[0.3cm]
  %
  \begin{enumerate}
    \item allow us to do analysis
    \item concept of limit is well behaved in $\R$
    \item (need more ... )
  \end{enumerate}
\end{frame}
% <==
% ==> rational converges to real
\section{Some important results}
\begin{frame}{Rational and real numbers}
  \begin{theorem}
    For any real number $x$, there is a sequence $(x_n)$ 
    of rationals such that $x_n\neq x$ for all $n\in\N$
    and \[\lim x_n=x.\]
  \end{theorem}

  To prove this theorem, we use the fact that $\Q$ is
  dense in $\R$. That is, for any two real numbers $x,y\in\R$
  where $x<y$, there exists a rational $r\in\Q$ such that
  $x<r<y$.
\end{frame}

\begin{frame}[b]
  \begin{proof}
    Let $x\in\R$. We wish to construct $(x_n)\subset\Q$ 
    such that $x_n\to x$.
    Because $\Q$ is dense in $\R$, then for each $n\in\N$
    we can find $x_n\in\Q$ such that $x<x_n<x+\frac{1}{n}$.\\[0.3cm]
    %
    We claim that $(x_n)$ does converge to $x$, and hence proved the
    theorem.\\[0.3cm]
    %
    Let $\epsilon>0$. Using Archimedean property, we can find $N\in\N$
    for which $N>\frac{1}{\epsilon}$. Now for any $n\geq N$, and because
    $x_n\in (x,x+\frac{1}{n})$ we then obtain that
    \[
      \abs{x_n-x}<\frac{1}{n}\leq\frac{1}{N}<\epsilon.
    \]
    Thus $x_n\to x$ as required.
  \end{proof}
\end{frame}


% <==




\end{document}
