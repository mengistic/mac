
% ==> mac template
%https://www.r-bloggers.com/2011/11/create-your-own-beamer-template/
%\usepackage{pgfcomp-version-0-65}
\usepackage{pgf}
\usepackage{tikz}
\usepackage{amsmath}
\linespread{1.25}
\usepackage{multicol}


% ==> math

\usepackage{mathtools}
\mathtoolsset{centercolon} % not work when using |mathpazo|
\DeclarePairedDelimiter\abs{\lvert}{\rvert}
\DeclarePairedDelimiterX\norm[1]\lVert\rVert{
	\ifblank{#1}{\:\cdot\:}{#1}
}
\def\set#1#2{\left\{#1 ~:~ #2\right\}}
\def\permil{\text{\hskip 0.3pt\englishfont\textperthousand}}


\DeclareMathOperator{\arccot}{cot}
\DeclareMathOperator{\arcsec}{arcsec}
\DeclareMathOperator{\arccsc}{arccsc}
\DeclareMathOperator{\lcm}{lcm}
\DeclareMathOperator{\ord}{ord}
\DeclareMathOperator{\sym}{sym}
\DeclareMathOperator{\tr}{trace}
\DeclareMathOperator{\spans}{span}
\DeclareMathOperator{\rank}{rank}
\DeclareMathOperator{\col}{col}
\DeclareMathOperator{\row}{row}
\DeclareMathOperator{\sign}{sign}
\DeclareMathOperator{\perm}{perm}
\DeclareMathOperator{\coef}{coef}
%\DeclareMathOperator{\proj}{proj}

\def\N{\mathbb N}
\def\Z{\mathbb Z}
\def\Q{\mathbb Q}
\def\R{\mathbb R}
\def\C{\mathbb C}
\def\F{\mathbb F}
\def\P{\mathbb P}

\def\L{\mathcal L}
\def\U{\mathcal U}

\def\labelitemi{$\circ$}
\def\inv{^{-1}}
\def\ang#1{\left\langle#1\right\rangle}
\def\tran{^\mathrm{T}}
\renewcommand{\vec}[1]{\mathbf{#1}}
\def\perc{^\perp}
\def\proj#1#2{\mathrm{proj}_{#2}{#1}}



\usepackage{bbold}
\let\altmathbb\mathbb
\AtBeginDocument{\let\mathbb\altmathbb}
% <==
% ==> color
\usepackage{color}

\definecolor{Ftitle}{RGB}{20, 0, 200}
\definecolor{Ftext}{RGB}{0,0,0}
\definecolor{Fenum}{RGB}{20, 70, 100}

\definecolor{StdTitle}{RGB}{26, 33, 141}
\definecolor{Descitem}{RGB}{20, 70, 100}
%\definecolor{StdTitle}{RGB}{20, 70, 100}
\definecolor{StdBody}{RGB}{213,24,0}

\definecolor{AlTitle}{RGB}{255, 190, 190}
\definecolor{AlBody}{RGB}{213,24,0}

\definecolor{ExTitle}{RGB}{201, 217, 217}
\definecolor{ExBody}{RGB}{213,24,0}


\setbeamercolor{normal text}{fg = Ftext}
\setbeamercolor{frametitle}{fg = Ftitle}
\setbeamercolor{title}{fg = Ftitle}
\setbeamercolor{section in toc}{fg = Ftitle}
\setbeamercolor{section in toc shaded}{fg = Ftitle}
\setbeamercolor{item}{fg = Fenum}
\setbeamercolor{subitem}{fg = Fenum}
\setbeamercolor{subsubitem}{fg = Ftitle}
\setbeamercolor{description item}{fg = Descitem}
\setbeamercolor{caption}{fg = Ftitle}
\setbeamercolor{caption name}{fg = Ftitle}

% Standard block
\setbeamercolor{block title}{fg = Descitem, bg = StdTitle!15!white}
\setbeamercolor{block body}{bg = StdBody!12!white}

% Alert block
\setbeamercolor{block title alerted}{bg = AlTitle}
\setbeamercolor{block body alerted}{bg = AlBody!5!white}

% Example block
\setbeamercolor{block title example}{bg = ExTitle}
\setbeamercolor{block body example}{bg = ExBody!5!white}

% <==
% ==> fonts
%\usefonttheme{professionalfonts}

% Font for the presentation title
\setbeamerfont{title}{size = \huge}

% Font of the frame titles
\setbeamerfont{frametitle}{size = \Large}
% <==
% ==> inner theme
\useinnertheme{circles}
\setbeamertemplate{itemize items}[circle]
\setbeamertemplate{enumerate items}[circle]
\setbeamertemplate{navigation symbols}{}

% ==> logo line
\newcommand{\LogoLine}{%
  \raisebox{-12mm}[0pt][0pt]{%
  \begin{pgfpicture}{0mm}{0mm}{0mm}{0mm}
    \pgfsetlinewidth{0.28mm}
    \color{gray!70!blue}
    \pgfline{\pgfpoint{-3 mm}{1mm}}{\pgfpoint{10.125 cm}{1mm}}
  \end{pgfpicture}}
}
\setbeamertemplate{headline}[text line]{\LogoLine}
% <==
% ==> logo
\setbeamertemplate{sidebar canvas right}{
  \vspace*{0 pt}\hspace*{-42 pt}%
  {\includegraphics[height=40 pt]{../pic/mac.jpg}}
}
% <==
% ==> adjust title
\setbeamertemplate{frametitle}{
  \vspace*{3mm}\hspace*{-2mm}\insertframetitle
}
% <==
% ==> footline
\newcommand{\Ffootline}{%
  \insertsection % The left end of the footline
  \hfill
  \textit{MAC} % The center
  \hfill
  \insertframenumber/\inserttotalframenumber % And the right end
}

\setbeamertemplate{footline}{%
  \usebeamerfont{structure}
  \begin{beamercolorbox}[
    wd=\paperwidth,ht=2.25ex,dp=1ex
    ]{title in head/foot}%
    \tiny\hspace*{4mm} \Ffootline \hspace{4mm}
  \end{beamercolorbox}
}
% <==
% <==
% ==> title
% Title

\newcommand{\TitleLine}{%
\raisebox{-12mm}[0pt][0pt]{%
\begin{pgfpicture}{0mm}{0mm}{0mm}{0mm}
\pgfsetlinewidth{0.10mm}
\color{gray}
\pgfline{\pgfpoint{55mm}{0mm}}{\pgfpoint{55mm}{50mm}}
\end{pgfpicture}}}

\newcommand{\MyTitle}{%
\hspace*{60mm}\vspace{-25mm}
\centering \inserttitle}

% Subtitle
\newcommand{\MySubTitle}{%
\hspace*{60mm}\vspace{-25mm}
\centering \footnotesize \textit{\insertsubtitle}}

% Author
\newcommand{\MyAuthor}{
\hspace*{60mm}\vspace{-25mm}
\centering \insertauthor}

% Institute
\newcommand{\MyInstitute}{
\hspace*{60mm}\vspace{-25mm}
\centering \footnotesize \textit{\insertinstitute}}

% Date
\newcommand{\MyDate}{
\hspace*{60mm}\vspace{-25mm}
\centering \insertdate}



% We declare the image that will be used as the logo
\pgfdeclareimage[width = 0.20\paperwidth]{big}{../pic/mac.jpg}

\setbeamertemplate{title page}{\TitleLine
\hspace*{11mm}\vspace*{-60mm}
\begin{beamercolorbox}[wd=0.5\paperwidth,ht=0.13\paperwidth]{Title}
\pgfuseimage{big}
\end{beamercolorbox}
%
\begin{beamercolorbox}[wd=\paperwidth,ht=0.06\paperwidth]{Title}
\usebeamerfont{Title}%
\MyTitle
\end{beamercolorbox}
%
\begin{beamercolorbox}[wd=\paperwidth,ht=0.03\paperwidth]{Title}
\usebeamerfont{Title}%
\MySubTitle
\end{beamercolorbox}
%
\begin{beamercolorbox}[wd=\paperwidth,ht=0.06\paperwidth]{Title}
\usebeamerfont{Title}%
\MyAuthor
\end{beamercolorbox}
%
\begin{beamercolorbox}[wd=\paperwidth,ht=0.03\paperwidth]{Title}
\usebeamerfont{head/foot}%
\MyInstitute
\end{beamercolorbox}
%
\begin{beamercolorbox}[wd=\paperwidth,ht=0.07\paperwidth]{Title}
\usebeamerfont{Title}%
\MyDate
\end{beamercolorbox}}
% <==
% <==
